% Options for packages loaded elsewhere
\PassOptionsToPackage{unicode}{hyperref}
\PassOptionsToPackage{hyphens}{url}
%
\documentclass[
]{article}
\usepackage{lmodern}
\usepackage{amssymb,amsmath}
\usepackage{ifxetex,ifluatex}
\ifnum 0\ifxetex 1\fi\ifluatex 1\fi=0 % if pdftex
  \usepackage[T1]{fontenc}
  \usepackage[utf8]{inputenc}
  \usepackage{textcomp} % provide euro and other symbols
\else % if luatex or xetex
  \usepackage{unicode-math}
  \defaultfontfeatures{Scale=MatchLowercase}
  \defaultfontfeatures[\rmfamily]{Ligatures=TeX,Scale=1}
\fi
% Use upquote if available, for straight quotes in verbatim environments
\IfFileExists{upquote.sty}{\usepackage{upquote}}{}
\IfFileExists{microtype.sty}{% use microtype if available
  \usepackage[]{microtype}
  \UseMicrotypeSet[protrusion]{basicmath} % disable protrusion for tt fonts
}{}
\makeatletter
\@ifundefined{KOMAClassName}{% if non-KOMA class
  \IfFileExists{parskip.sty}{%
    \usepackage{parskip}
  }{% else
    \setlength{\parindent}{0pt}
    \setlength{\parskip}{6pt plus 2pt minus 1pt}}
}{% if KOMA class
  \KOMAoptions{parskip=half}}
\makeatother
\usepackage{xcolor}
\IfFileExists{xurl.sty}{\usepackage{xurl}}{} % add URL line breaks if available
\IfFileExists{bookmark.sty}{\usepackage{bookmark}}{\usepackage{hyperref}}
\hypersetup{
  pdftitle={Does space use behavior relate to exploration in a species that is rapidly expanding its geographic range?},
  pdfauthor={McCune KB1*; Ross C2; Folsom M2; Bergeron L1; Logan CJ2},
  hidelinks,
  pdfcreator={LaTeX via pandoc}}
\urlstyle{same} % disable monospaced font for URLs
\usepackage[margin=1in]{geometry}
\usepackage{graphicx,grffile}
\makeatletter
\def\maxwidth{\ifdim\Gin@nat@width>\linewidth\linewidth\else\Gin@nat@width\fi}
\def\maxheight{\ifdim\Gin@nat@height>\textheight\textheight\else\Gin@nat@height\fi}
\makeatother
% Scale images if necessary, so that they will not overflow the page
% margins by default, and it is still possible to overwrite the defaults
% using explicit options in \includegraphics[width, height, ...]{}
\setkeys{Gin}{width=\maxwidth,height=\maxheight,keepaspectratio}
% Set default figure placement to htbp
\makeatletter
\def\fps@figure{htbp}
\makeatother
\setlength{\emergencystretch}{3em} % prevent overfull lines
\providecommand{\tightlist}{%
  \setlength{\itemsep}{0pt}\setlength{\parskip}{0pt}}
\setcounter{secnumdepth}{-\maxdimen} % remove section numbering
\usepackage[left]{lineno}
\linenumbers

\title{Does space use behavior relate to exploration in a species that is
rapidly expanding its geographic range?}
\author{\href{https://www.kelseymccune.com/}{McCune KB}\textsuperscript{1}* \and Ross C\textsuperscript{2} \and Folsom M\textsuperscript{2} \and Bergeron L\textsuperscript{1} \and \href{http://CorinaLogan.com}{Logan CJ}\textsuperscript{2}}
\date{2020-11-10}

\begin{document}
\maketitle

\hypertarget{affiliations}{%
\subparagraph{Affiliations:}\label{affiliations}}

\begin{enumerate}
\def\labelenumi{\arabic{enumi})}
\tightlist
\item
  University of California Santa Barbara
\item
  Max Planck Institute for Evolutionary Anthropology
\end{enumerate}

*Corresponding author: KB McCune
(\href{mailto:kelseybmccune@gmail.com}{\nolinkurl{kelseybmccune@gmail.com}})

\textbf{Cite as:} McCune K, Ross C, Folsom M, Bergeron L, Logan CJ.
2020. \href{http://corinalogan.com/Preregistrations/gspaceuse.html}{Does
space use behavior relate to exploration in a species that is rapidly
expanding its geographic range?}
(\url{http://corinalogan.com/Preregistrations/gspaceuse.html}) In
principle acceptance by \emph{PCI Ecology} of the version on 23 Sep 2020
\url{https://github.com/corinalogan/grackles/blob/master/Files/Preregistrations/gspaceuse.Rmd}.

\textbf{This preregistration has been pre-study peer reviewed and
received an In Principle Recommendation by:}

Blandine Doligez (2020) Explore and move: a key to success in a changing
world? \emph{Peer Community in Ecology}, 100058.
\href{https://doi.org/10.24072/pci.ecology.100058}{10.24072/pci.ecology.100058}

\begin{itemize}
\tightlist
\item
  Reviewers: Joe Nocera, Marion Nicolaus, and Laure Cauchard
\end{itemize}

\hypertarget{abstract}{%
\subsubsection{ABSTRACT}\label{abstract}}

Great-tailed grackles (\emph{Quiscalus mexicanus}) are rapidly expanding
their geographic range (Wehtje 2003). Range expansion could be
facilitated by consistent behavioural differences between individuals on
the range edge and those in other parts of the range (Duckworth and
Badyaev 2007; Lindström et al. 2013). Movement behaviors in particular
are thought to be important for expanding and invasive species (Bubb et
al. 2006). There is evidence for a relationship between individual
exploratory traits and dispersal (the permanent movement an individual
makes from its birth site to the place where it reproduces; Greenwood
and Harvey 1982), but it is still unknown whether individual differences
in exploration relate to adult daily movement patterns within the home
range (``space use''). Evidence suggests that daily space use behavior
can vary consistently among individuals (Hertel et al. 2020), but no
studies have examined the relationship between space use behavior and
measures of exploration, or space use behavior in different populations
along an expanding range. Here we will study grackles from 3 populations
that span the extent of the current range - Central America (their
original range), Arizona (middle of the northern expanding edge), and
northern California (near the northern edge of their range). We will
test whether performance on an exploration task in captivity relates to
subsequent space use behavior in the wild measured using home range
size, the autocorrelation of step length (distance between two
sequential observations) and turning angle for each individual over time
(Pacheco-Cobos et al. 2019), and the repeatability of each individual's
occurrence in particular geographic locations. Results will inform
whether individual differences in space use behavior are associated with
consistent individual differences in exploration, and whether daily
space use patterns differ in populations at different points on an
expanding range. If space use behavior correlates with measures of
exploration, then space use data could be used to inform conservation
management strategies (e.g.~identify which individuals are likely to
remain in new or restored habitat after a translocation (May et al.
2016)) in species where it is not logistically feasible to \emph{a
priori} measure exploration in captivity. If daily space use patterns
differ in populations at different locations on the expanding range,
then future studies could investigate whether space use and exploration
are coupled with other traits such as long-distance natal dispersal to
form an ``invasion syndrome'' (Chapple et al. 2012).

\hypertarget{introduction}{%
\subsubsection{INTRODUCTION}\label{introduction}}

Problematic range expansions of invasive species are occurring across
the globe. However, not all species that invade novel areas outside of
their traditional range become established (Hayes and Barry 2008;
Chapple et al. 2012) and it is still unclear what characteristics
facilitate successful invasions (Fogarty et al. 2011). These
characteristics are important for predicting potential invasions,
especially since invasive species are implicated as a leading cause of
biodiversity loss (Clavero and Garcı́a-Berthou 2005; Butchart et al.
2010). One hypothesis is that the composition of behavioral types might
facilitate geographic range expansion (Carere and Gherardi 2013) because
certain individuals may possess traits such as exploration and
aggressiveness that make them more likely to succeed when venturing into
new habitats and competing with heterospecifics (Cote et al. 2010).
Evidence is already accumulating that links consistent individual
differences in behavioral types and invasion success. For example,
Duckworth and Badyaev (2007) found more aggressive behavior in western
bluebird males on the range edge, suggesting that range expansion was
facilitated by aggressive males dispersing further and displacing less
aggressive mountain bluebirds. Additionally, invasive cane toads on the
invasion front in Australia spent longer in the dispersal mode and moved
greater distances than toads tracked several years later in the same
location (Lindström et al. 2013).

Within-species variation in the ability (movement) and motivation
(exploratory tendency) to encounter conspecifics, novel foods, and novel
food sources could be a limiting factor in successful species range
expansions (Spiegel and Crofoot 2016). In novel areas, the occurrence of
conspecifics, food, predators and other environmental factors may not be
as easily detectable or recognizable, and may be distributed differently
across the landscape than in core areas of the range. Although
individuals with exploratory phenotypes may be more successful at
colonizing new habitat, exploratory individuals are also at higher risk
of predation (Stuber et al. 2013), and could be less likely to find
local food sources (Overveld and Matthysen 2009). Consequently, for
establishment in new areas, individuals that exhibit a range of
exploratory behavior are needed, and the interaction between space use
and exploratory tendency is likely important for finding novel foods and
food sources. Additionally, while dispersal (the permanent movement an
individual makes from its birth site to the place where it reproduces;
Greenwood and Harvey 1982) is necessary to initially invade the novel
habitat, subsequent daily space use could determine establishment
success. For example, on the range edge conspecific density might be
lower and individuals may need to use space differently (Bubb et al.
2006). However, current research on invasion success and movement
behavior focuses heavily on dispersal, and while dispersal and
exploratory tendencies have been shown to be associated (Cote et al.
2010), we do not know how exploratory tendency influences space use
patterns in the daily lives of invading individuals.

Space use behavior is expected to be influenced by internal states like
exploratory tendency and hunger, as well as the non-random distribution
of available habitat and resources (Nathan et al. 2008). Space use can
also consistently differ among individuals (Hertel et al. 2020), which
indicates that each individual has distinct preferences for how, when,
and where to move within its home range. Traditional analyses of animal
space use required spatial and temporal independence of data points for
statistical analysis (Swihart and Slade 1985), yet movement data are
unlikely to meet these criteria. Spatial and temporal autocorrelation
occurs when the position of an individual at a given time is tightly
linked to its position both before and after, including cases where
individuals are repeatedly found in the same locations across time. This
autocorrelation is an intrinsic component of space use behavior and
eliminating it can reduce biological relevance and obscure relationships
with behavioral types (Dray et al. 2010). Therefore, the autocorrelated
nature of movement paths could be important to illuminate the
relationship between individual differences in exploratory tendency and
daily space use.

Great-tailed grackles (\emph{Quiscalus mexicanus}, hereafter
``grackles'') are rapidly expanding their geographic range (Wehtje
2003). They are invasive because they meet the criteria for the
establishment and spread stages of invasion (Blackburn et al. 2011), and
they are considered a pest in some areas where they reduce fruit crop
yields (Glahn et al. 1997). The nature and level of ecological and
social factors grackles experience may vary in importance between
populations. Generally, this species is strongly associated with
human-modified landscapes and is able to take advantage of a variety of
human foods (e.g.~crops, at our outdoor cafes, and out of our garbage
cans) in addition to foraging on insects and on the ground for natural
food items (Johnson and Peer 2001). Furthermore, they exhibit variation
in territorial social behavior in both the breeding and non-breeding
seasons. During the non-breeding season, this species forages in smaller
groups and communally roosts in larger groups (Johnson and Peer 2001).
During the breeding season, one or more males defend a territory where
multiple females place their nests within that territory to raise the
young (Johnson et al. 2000). Roaming males are also present and can
obtain extra-pair copulations with females on other males' territories
(Johnson et al. 2000).

In this investigation, we aim to understand whether exploratory
tendency, measured in captivity, is associated with space use behavior
measured in the wild in grackles from three populations that span the
current range: Central America (their original range), Arizona (middle
of the northern expanding edge), and northern California (near the
northern edge of their range). Exploration, measured here following the
protocol described in
\href{http://corinalogan.com/Preregistrations/g_exploration.html}{McCune
et al.~2019}, is interpreted as an individual's response to novelty,
such as novel environments or novel objects (Réale et al. 2007), to
gather information that does not satisfy immediate needs (Mettke-Hofmann
et al. 2002). Previous studies on birds have found a relationship
between some movement behaviors (home range size and natal dispersal) in
the wild and exploration measured in captivity using novel environment
(Dingemanse et al. 2003; Minderman et al. 2010) and novel object
(Mettke-Hofmann et al. 2002) tests. Therefore, these measures have the
potential to be relevant to grackles.

After measuring exploratory tendency in captivity, we will release
grackles and use telemetry to measure their space use behavior. We will
quantify a typical measure of space use that controls for
autocorrelation (i.e.~home range size), but we will also investigate the
behavioral relevance of autocorrelation in space use with two relatively
new methods. The first method will describe individual differences in
path-level movement behavior by analyzing the autocorrelation of step
length (distance between two sequential observations) and turning angle
for each individual over time (Pacheco-Cobos et al. 2019; Hertel et al.
2020), while the second method will describe individual differences in
spatial preferences by analyzing the repeatability of each individual's
occurrence in particular geographic locations.

With these data we will test two separate hypotheses. First, we will
test whether grackles' performance on exploration tasks in captivity is
related to the space use metrics of the same individuals that we
quantify from their movement behavior in the wild. Second, we will test
the hypothesis that the space use behavior of wild grackles
systematically varies with population location along the extent of the
range. These results will have implications for invasive species
management (Carere and Gherardi 2013), but also could be used to inform
the conservation of threatened and endangered species (e.g.~identify
which individuals are likely to remain in new or restored habitat after
a translocation; May et al. 2016). This will be especially relevant for
species where it is not logistically feasible to \emph{a priori} measure
exploration in captivity.

\hypertarget{a.-state-of-the-data}{%
\subsubsection{A. STATE OF THE DATA}\label{a.-state-of-the-data}}

This preregistration uses secondary data: data that are already being
collected for other purposes (GPS points in hypothesis 3 and home range
sizes in prediction 3 in the
\href{http://corinalogan.com/Preregistrations/g_flexforaging.html}{flexibility
and foraging} preregistration). Originally we attached radio tags to
grackles released from the aviaries to ensure that we could find their
nest sites and track measures of foraging behavior and reproductive
success for these individuals for which we have an extensive amount of
behavioral data from aviary tests. Now we plan to additionally use the
radio tags to collect data for this space use preregistration. This
preregistration was written in June 2019, while at the same time
increasing the number of GPS points taken per tracking session per bird
to provide enough data for the analyses here, and submitted in September
2019 to PCI Ecology for pre-study peer review. Reviews were received in
December 2019 and we revised and resubmitted in March 2020. A second
round of reviews was received in July 2020 and we revised and
resubmitted in August 2020. A third round of reviews was received on
September 1, 2020 and we revised and resubmitted on September 15, 2020.
A fourth round of reviews was received on September 17, 2020 and we
revised and resubmitted on September 30, 2020.

\hypertarget{b.-hypotheses}{%
\subsubsection{B. HYPOTHESES}\label{b.-hypotheses}}

\textbf{H1: Individual differences in measures of exploration using
novel environment and novel object tasks (see
\url{http://corinalogan.com/Preregistrations/g_exploration.html} for
methods{]} are related to variation in space use (measured via home
range size, autocorrelation of step lengths and turning angles, or
whether individuals are predictably found in the same locations) across
the breeding and non-breeding seasons. Previous studies on birds have
found a relationship between movement behavior in the wild and
exploration measured in captivity using novel environment (Dingemanse et
al. 2003; Minderman et al. 2010) and novel object (Mettke-Hofmann et al.
2002) tests, therefore the measures we are investigating have the
potential to be relevant to grackles. We expect space use to vary within
an individual across breeding seasons because during the non-breeding
season this species forages in smaller groups and communally roosts in
larger groups (Johnson and Peer 2001). During the breeding season, one
or more males defend a territory and females place their nests within
territories to raise the young (Johnson et al. 2000). Roaming males are
also present and can obtain extra-pair copulations with females on other
male's territories (Johnson et al. 2000).}

\textbf{Prediction 1:} More exploratory grackles, i.e.~individuals that
get closer to the novel object and novel environment will be found in a
larger expanse (larger home range size), use less predictable movement
patterns (low autocorrelation of step lengths and turning angles), and
occupy a greater variety of spatial locations. This would suggest that
exploratory individuals are more willing to move into novel areas in the
wild.

\textbf{Prediction 1 alternative 1:} The more exploratory grackles will
be found in a smaller expanse (smaller home range size), use more
predictable movement patterns (high autocorrelation of step lengths and
turning angles), and consistently occupy the same spatial locations.
This would suggest that the more exploratory individuals may dedicate
more time to investigating a smaller area within their home range rather
than moving into new areas for resources such as food or mating
opportunities.

\textbf{Prediction 1 alternative 2:} Only performance on the novel
environment task will correlate positively with space use behavior in
the wild. This would suggest that perception of, and behavioral
interactions with, novel environments (spatial information) differs from
that used for novel objects (Mettke-Hofmann et al. 2009).

\textbf{Prediction 1 alternative 3:} Only performance on the novel
object task will correlate positively with space use behavior in the
wild. This would suggest that, in these populations located in
human-modified environments, space use may primarily be driven by
grackles searching for novel objects that represent human-provided
sources of food. Much of the food grackles consume is contained within
human-made packaging (e.g.~grackles search inside take out bags from
restaurants) or enclosed in human-made containers (e.g.~garbage cans),
therefore they should have a reason to approach and explore new objects
to determine whether they could be a new food source.

\textbf{Prediction 1 alternative 4:} There will be no correlation
between an individual's proximity or touches to the novel object or
novel environment and their space use behavior. This would suggest that
the measures of exploration in captivity either are not relevant enough
to how grackles use space in the wild to be able to measure the same
trait, or they are independent of space use behavior potentially because
the individuals tracked are primarily adults and are already familiar
with their home range and surrounding areas and thus do not need to
further use the space as if it were novel.

\textbf{H2: Space use behavior will vary among grackles from our three
study populations located along different points in the geographic range
of this species (core, middle of expansion, and range edge). These
populations are theoretically connected, however actually moving between
two of our field sites within a few grackle lifespans is unlikely due to
the large distances between field sites and two geographic barriers (the
Sierra Nevada and Sierra Madre mountain ranges, and the high elevation
areas of Mexico).}

\textbf{Prediction 2:} Home range size will increase, autocorrelation of
step lengths and turning angles will decrease (i.e.~grackle movement
behavior will be less predictable), and grackles will use a greater
variety of spatial locations as the geographic distance from the
original center of the range increases. Specifically, the grackles
sampled from our site on the edge of the range (northern California),
will have larger overall home range sizes, exhibit more variety in step
lengths and turning angles, and use a greater variety of spatial
locations than the sample of grackles in the core of the range (Central
America). Grackles in the sample from the middle of the expanding range
will be intermediate in space use (Arizona). Such population differences
in space use behavior may relate to range expansion because some of the
individuals on the leading edge of the range may use more space and move
longer distances (Duckworth and Badyaev 2007). However, larger-scale
sampling of grackle groups across the strata of the expansion front and
core range would be needed to more robustly validate the hypothesis that
our cross-site differences are indicative of a broader pattern driven by
the location of the expansion front.

\textbf{Prediction 2 alternative 1:} Grackles sampled on the edge of the
range will have smaller overall home range sizes, high autocorrelation
in step length and turning angle (i.e.~movement behavior will be more
predictable), and consistently use the same spatial locations compared
to grackles sampled in the middle or core of the current range. This
would suggest that suitable habitat may be distributed in small patches,
that novel habitats at the edge of the range may have high predation on
grackles that use more space, and/or that individual grackles specialize
on certain novel habitat types that are patchily distributed.

\textbf{Prediction 2 alternative 2:} We will find no difference across
the geographic range in the space use behavior of the grackles sampled.
This would suggest that, on average, all grackles may use the same
amount of space, or that there is a similar distribution of individual
differences in space use in each population. Alternatively, it could
indicate that we did not detect differences because we measured adults
rather than juveniles. Grackles sampled in different populations may
converge on similar space use behavior during development, or juvenile
grackles may disperse further on the edge of the range. However, we are
not able to detect these differences with our data, which is primarily
from adults.

\hypertarget{c.-methods}{%
\subsubsection{C. METHODS}\label{c.-methods}}

\hypertarget{planned-sample}{%
\paragraph{Planned Sample}\label{planned-sample}}

Great-tailed grackles are caught in the wild, given colored leg bands in
unique combinations for individual identification, and released at their
point of capture. The color-marked grackles in this study have one of
two different backgrounds: those that do not have radio tags and those
that do. First, we opportunistically track color-marked grackles that do
not have a radio tag (and thus have not spent time in the aviaries) to
compare whether time spent in the aviaries is related to space use
behavior. When a color-marked bird is encountered, researchers track it
for 20-90 minutes, recording the spatial location every one minute. If
the bird goes out of view, researchers attempt to find it again for
15-30 minutes before moving on. Because these data are opportunistic, we
do not attempt to balance for sex, but we aim to follow at least 20
non-tagged individuals in each population.

Second, we applied radio tags to all aviary-tested birds (estimated 20
individuals per population). These subjects are primarily adults who we
attempt to balance for sex, and who spent up to six months in an aviary
while they participated in behavioral choice tests (see
\href{http://corinalogan.com/Preregistrations/g_flexmanip.html}{Logan et
al.~2019} for details) and individual differences assays, including
measures of exploration in captivity (see
\href{http://corinalogan.com/Preregistrations/g_exploration.html}{McCune
et al.~2019} for details), as part of other research projects by this
lab. For details about the captive environment, please refer to the
preregistration associated with this part of the research:
\href{http://corinalogan.com/Preregistrations/g_exploration.html}{McCune
et al.~2019}. Cognitive and behavioral traits are often not fully
developed in hatch year birds (i.e.~Zucca et al. 2007). For our aviary
cognitive test battery, we avoided the potentially confounding variable
of stage of cognitive development by only taking known adult grackles
into the aviaries. We identified grackles as adults using eye color,
where hatch year birds have brown eyes, but second year and older birds
have yellow eyes. While it is possible that cognitive traits continue to
develop in second year and older birds, it is impossible to distinguish
grackle age after the bird's first year (Johnson and Peer 2001).

Before the aviary grackles are released, they are fitted with VHF radio
tags (Lotek PipLL (model Ag386), Advanced Telemetry Systems (model
A2455) or Holohil Systems Ltd.~(model BD-2)) so we can track space use
behavior using radio telemetry. Radio tags were initially attached to
the grackles by gluing them to their backs (Johnson and Peer 2001; Mong
and Sandercock 2007), however these did not stay on for very long.
Therefore, we now use a leg loop harness (methods as in Rappole and
Tipton 1991) made from sutures and secured with crimp beads and
cyanoacrylate glue (Vicryl undyed 36in sutures, item number D9389 at
eSutures.com; 0.5mm diameter, absorbable so they fall off after one to
four months).

After release, an experimenter will find and follow (each session is
called a ``track'') each tagged grackle at least four times per week. On
each of these tracks, the experimenter follows the focal individual for
approximately 1.5 hours, recording a GPS point every one minute,
regardless of whether the bird moved (Cushman et al. 2005).
Additionally, we aim to balance tracking data equally during morning and
afternoon time periods for all grackles. We hoped to track the same
individuals across the breeding and nonbreeding seasons. However, the
leg loop harnesses often degraded and fell off after 1 - 4 months.
Therefore, we have few data points on the same individuals in both
seasons and we will instead compare space use in different individuals
(rather than within individuals) across seasons. Researchers maintain a
distance of at least 30 m and observe the bird with binoculars so the
grackle's behavior is not influenced or artificially changed. If the
grackle alarm calls while oriented towards the researcher (indicating
the researcher's presence affected the grackle's behavior), all tracking
on that individual is stopped for the day. To ensure we capture all
locations the individual visits and not just those where they are most
easily seen and followed, tagged grackles that move out of sight during
tracking are searched for with telemetry until they are found again.

Exploratory tendency is an intrinsic factor that could be related to
space use behaviors. However, there are also likely alternative
variables that may relate to space use behavior in wild grackles that we
must control for by including them as covariates in our models. First,
we measure energetic condition (described in
\href{http://corinalogan.com/Preregistrations/gcondition.html}{Berens et
al.~2019}) to account for differences in the physiological mobility that
may limit an individual's space use behaviors (Nathan et al. 2008).
Secondly, we measure habitat characteristics such as human food sources
and available breeding habitat (described in
\href{http://corinalogan.com/Preregistrations/g_flexforaging.html}{Logan
et al.~2019}) because these factors of the external environment will
affect where grackles choose to move or spend time (Nathan et al. 2008).

Conspecific density has also been shown to affect home range size in
other bird species (Flockhart et al. 2016; Garabedian et al. 2018). To
control for the possibility that home range size may vary among our
populations due to conspecific density rather than exploratory traits,
we will use point count surveys to measure grackle population density.
We will place 225 point count stations across the landscape encompassing
each population (Tempe, AZ; Woodland, CA; Gamboa, Panama). For each
study population, the first central point will be randomly placed within
500 m of the center of the study area. The remaining points are placed
in a 500 m grid pattern extending out from this central point. In total,
the sample area will cover an area that is 7 km by 7 km. Each point will
be visited once during the non-breeding season (Sep-Mar). During the
survey, researchers will record all grackles visually and aurally
detected for six minutes.

\hypertarget{sample-size-rationale}{%
\paragraph{Sample size rationale}\label{sample-size-rationale}}

We test as many birds as we can during the approximate five years of
this study given that the birds are only brought into the aviaries
during the non-breeding season (approximately September through March).
It is time intensive to conduct the aviary test battery (2-6 months per
bird at the Arizona field site), therefore we approximate that the
minimum sample size for captive subjects will be 60 across the three
sites (approximately 20 birds per site with the aim that half of the
grackles tested at each site are females). We catch grackles with a
variety of methods (mist nets, walk-in traps, and bow nets), some of
which decrease the likelihood of a selection bias for exploratory and
bold individuals because grackles cannot see the traps (i.e.~mist nets).
Once released, we will primarily track the space use behavior of these
\textasciitilde60 grackles that have radio tags. As described above, we
will also opportunistically collect GPS point locations non-tagged
grackles to determine whether grackles that were previously in the
aviary have different space use behavior from non-aviary-held grackles
after their release. We will attempt to match the sample size of aviary
birds, and in our Arizona population we currently have over 20 points
(the minimum number for reliably calculating home range size; Noonan et
al. 2019) for 33 individuals that have never had radio tags. We aim to
acquire more than 20 points on at least 20 non-tagged grackles in the
other two populations as well. Additionally, we attach radio tags to
birds that are released early because of their lack of willingness to
participate in aviary tests (currently 5 individuals) to determine
whether space use behavior differs between participatory and
non-participatory grackles.

\textbf{Data collection stopping rule}

We will stop collecting GPS location data on tagged and non-tagged birds
when home ranges are fully revealed for data collected in both breeding
and non-breeding seasons. To determine at what point home ranges have
been fully revealed, we will calculate the asymptotic convergence of
home range area as in Leo et al. (2016). We will test home range
asymptotic convergence for breeding season and non-breeding season
movements separately (breeding season: Apr - Aug, non-breeding season:
Sep - Mar).

\textbf{Open materials}

Protocols:

\begin{itemize}
\tightlist
\item
  Exploration protocol for exploration of new environments and objects,
  boldness, persistence, and motor diversity:
  \url{https://docs.google.com/document/d/1sEMc5z2fw6S9C-wVfc2zV331CRPpu3NuA7IhSFUZJpE/edit?usp=sharing}
\item
  Radio tracking protocol for attaching radio tags and collecting GPS
  points using radio telemetry:
  \url{https://docs.google.com/document/d/1jtjgeWJoZ0Q1CfUpV6zdkyQL3p3WfW9KgyLrMNmNMJc/edit?usp=sharing}
\item
  Point count protocol for measuring grackle population density in the
  study area:
  \url{https://docs.google.com/document/d/1zDuoI0v7Rv0iwrTMAZSERH7AB4v85qXzxsx5T0daNNo/edit?usp=sharing}
\end{itemize}

\textbf{Open data}

When the study is complete, the data will be published in the Knowledge
Network for Biocomplexity's data repository.

\textbf{Randomization and counterbalancing}

There is no randomization in this investigation. The order of the
exploration tasks is counterbalanced across birds (see the separate
preregistration -
\url{http://corinalogan.com/Preregistrations/g_exploration.html} for
details). The time of day that we collect GPS point locations is
counterbalanced within and across birds to account for potential
variation in movement behavior arising from daily circadian rhythms.

\textbf{Blinding of conditions during analysis}

No blinding is involved in this investigation.

\hypertarget{summary-of-methods-for-measuring-exploration-mccune-et-al.-2019---httpcorinalogan.compreregistrationsg_exploration.html}{%
\subsubsection{\texorpdfstring{Summary of methods for measuring
exploration (McCune et al.~2019 -
\url{http://corinalogan.com/Preregistrations/g_exploration.html})}{Summary of methods for measuring exploration (McCune et al.~2019 - http://corinalogan.com/Preregistrations/g\_exploration.html)}}\label{summary-of-methods-for-measuring-exploration-mccune-et-al.-2019---httpcorinalogan.compreregistrationsg_exploration.html}}

We adapted commonly used methods to test exploratory tendency of the
grackles that are temporarily held in aviaries in response to a novel
environment and a novel object. Exploratory tendency is measured as the
latency to approach and the closest approach distance to a novel object
and a novel environment. Exploration assays occur twice for each bird:
once near the beginning of their aviary time (``time 1'') and once again
approximately 6 weeks later (``time 2''). Habituation may occur between
time 1 and time 2, decreasing the novelty of the experimental setup.
However, it is common practice to use the same setup across the repeated
assays because it is very difficult to predict how threatening a novel
object will be to a grackle. Therefore, if we accidentally introduce
objects that are much more or much less threatening across the two time
periods, this could obscure our ability to determine whether there are
consistent individual differences with regard to these particular novel
objects. We will analyze whether behavioral responses during assays are
repeatable within individuals and whether exploration of a novel
environment correlates with exploration of a novel object, indicating
they are measures of the same inherent trait. If the two exploration
measures are consistent within individuals and correlate with each
other, we will choose as the exploration score the variable with the
most data. If the two measures do not correlate, we will include both as
independent variables.

\textbf{Dependent variables}

\textbf{P1-P2}

\begin{enumerate}
\def\labelenumi{\arabic{enumi})}
\item
  Home range size (square meters): an estimate calculated using the
  autocorrelated-Gaussian reference function kernel density estimate
  (AKDE), which is the only estimate of home range that accounts for
  autocorrelation due to the small time period between each of our GPS
  locations (Noonan et al. 2019). This estimate consists of the area
  enclosing the GPS location points for an individual grackle during its
  normal activities.
\item
  Autocorrelation of step length (meters): measured as the standard
  deviation of step lengths (the distance between two sequential GPS
  points)
\item
  Autocorrelation of turning angle (degrees): measured as the standard
  deviation of turning angles
\item
  Spatial location preference: measured as the repeatability of grackle
  occurrence in a given cell of a 5 x 5 m grid array across the
  landscape
\end{enumerate}

One model will be run for each dependent variable

\textbf{Independent variables}

\textbf{P1 and P1 alternatives 1-4}

\begin{enumerate}
\def\labelenumi{\arabic{enumi})}
\item
  Exploration of novel environment: Latency to approach up to 20cm of a
  novel environment (that does not contain food) set inside a familiar
  environment (that contains maintenance diet away from the object) - OR
  - closest approach distance to the novel environment (choose the
  variable with the most data)
\item
  Exploration of novel object: Latency to approach up to 20cm of an
  object (novel or familiar, that does not contain food) in a familiar
  environment (that contains maintenance diet away from the object) - OR
  - closest approach distance to the object (choose the variable with
  the most data)
\item
  Sex: Male or female
\item
  History: the number of days the individual was temporarily held in the
  aviaries before data collection on space use began (0 indicates the
  grackle was only ever in the wild)
\item
  The number of known breeding sites (shade trees, date palms, marsh
  vegetation (Johnson and Peer 2001)) within the home range of each
  individual (data collected as part of Logan et al.~2019 -
  \url{http://corinalogan.com/Preregistrations/g_flexforaging.html})
\item
  The number of human food source areas (dumpsters, cat food bowls,
  outdoor restaurant seating areas and parking lots) within the home
  range of each individual (data collected as part of
  \href{http://corinalogan.com/Preregistrations/g_flexforaging.html}{Logan
  et al.~2019})
\item
  Scaled mass index (Peig and Green 2009) as a measure of energetic
  condition
\item
  Maximum group size observed across each individual's focal follows
  (data collected as part of Logan et al.~2019 -
  \url{http://corinalogan.com/Preregistrations/g_flexforaging.html})
\item
  Season (breeding or non-breeding)
\end{enumerate}

\textbf{P2}

\begin{enumerate}
\def\labelenumi{\arabic{enumi})}
\item
  Site: Whether the grackle was from our study population located on the
  edge of the range (Northern California), the center of the original
  range (Central America), or the center of the current expanding edge
  (Arizona).
\item
  Sex: Male or female
\item
  History: the number of days the individual was temporarily held in the
  aviaries before data collection on space use began (0 indicates the
  grackle was only ever in the wild)
\item
  Population density (number of grackles per square meter in each study
  area: Arizona, California, Central America)
\item
  Season (breeding or non-breeding)
\end{enumerate}

\hypertarget{d.-analysis-plan}{%
\subsubsection{D. ANALYSIS PLAN}\label{d.-analysis-plan}}

We do not plan to \textbf{exclude} any data and if there are
\textbf{missing} data (e.g.~if a bird had to be released before
collecting their data at time 2) these birds will be excluded from
analyses requiring data from times 1 and 2. Analyses will be conducted
in R (current version 4.0.2; R Core Team 2017) and Stan (version 2.18;
Carpenter et al. 2017).

When calculating our dependent variables we are interested in all
movements by grackles, therefore we will not exclude any outlier
relocations collected during ``normal daily activities'' (Calenge 2011).
``Normal daily activities'' indicate that grackles are not engaging in
behaviors that would artificially skew their space use, for example
mobbing a predator or the experimenter, or behavior before sunrise or
after sunset when they are at the roost. Outside of these circumstances,
we will include all data to detect space use movements.

From the GPS point locations collected on each individual, we will
calculate home range size using the akde function in the R packages ctmm
(Calabrese et al. 2016) and sf (Pebesma 2018). We will use a Bayesian
model (detailed below) to estimate the following parameters: mean and
dispersion (variance) of step lengths and turning angles for each bird
on each daily track (Pacheco-Cobos et al. 2019). We will determine
whether these parameters governing movement are stable or variable
within individuals across days. A small variance would indicate there is
low variability (high repeatability) in the daily movement behaviors of
the individual. Moreover, we will determine whether grackles show
individual differences in consistent use of habitat by overlaying a grid
array across the landscape. We will then create matrices describing the
number of times a grackle was observed in each cell on each day. High
autocorrelation among daily matrices indicates an individual that
frequents the same spatial locations across days.

We will quantify region-specific grackle density by fitting a
hierarchical model that accounts for imperfect detection. Specifically,
we will use the model developed by Amundson et al. (2014) which
integrates data on time of detection and distance estimates to account
for the probability a bird is available to be detected (pa), and the
probability it is detected given it is available (pd), respectively
(Nichols et al. 2009). All parameters (density, pa, and pd) will be
modeled as a function of region. Because we expect that vocalization
rates will be greater for males than females, we will also model pa as a
function of sex. We will extract the estimate of expected point-level
density for each region and use this estimate as the covariate in our
movement models.

We will then model the relationship between independent variables
describing bird-specific data on performance in the exploration tasks
and other covariates (as outlined above), and bird-specific movement
parameters (e.g.~home range size, autocorrelation in step-size and
turning angle, repeatability of spatial location preferences) as our
dependent variables. We will use linear models and we will ensure
assumptions of normality are met by checking that the residuals from our
fitted models are normally distributed.

We will additionally test whether time spent in captivity might alter
the social behavior of grackles when subsequently released into the wild
by testing maximum group size observed across each individual's focal
follows as a function of the individual's captivity history (the number
of days the individual was temporarily held in the aviaries before data
collection on space use began).

\hypertarget{ability-to-detect-actual-effects}{%
\paragraph{Ability to detect actual
effects}\label{ability-to-detect-actual-effects}}

To assess the effects that we will be able to detect given the expected
sample size, we used G*Power (v.3.1; Faul et al. 2007; Faul et al. 2009)
to conduct power analyses based on confidence intervals. G*Power uses
pre-set drop down menus and we chose the options that were as close to
our analysis methods as possible (listed in each analysis below). These
power analyses are not fully aligned with our study design, and the
expected effect sizes are difficult to estimate due to the lack of prior
data on this species; yet we are unaware of better options at this time.

\hypertarget{calculating-home-range-size}{%
\paragraph{Calculating home range
size}\label{calculating-home-range-size}}

\emph{code not shown in pdf, see .rmd link above}

\hypertarget{code-to-create-functions-for-analyzing-movement-behaviors}{%
\paragraph{Code to create functions for analyzing movement
behaviors}\label{code-to-create-functions-for-analyzing-movement-behaviors}}

All scripts and code are available at
\url{https://github.com/ctross/grackleator}. \emph{code not shown in
pdf, see .rmd link above}

\hypertarget{modeling-bird-movement-behaviors-step-length-turning-angle-spatial-location-preference}{%
\paragraph{Modeling bird movement behaviors: step length, turning angle,
spatial location
preference}\label{modeling-bird-movement-behaviors-step-length-turning-angle-spatial-location-preference}}

\emph{code not shown in pdf, see .rmd link above}

\textbf{H1: P1 - Exploration measured in captivity relates to space use
behavior}

To roughly estimate our ability to detect actual effects, we ran a power
analysis in G*Power with the following settings: test family=F tests,
statistical test=linear multiple regression: Fixed model (R\^{}2
deviation from zero), type of power analysis=a priori, alpha error
probability=0.05. We reduced the power to 0.70 and increased the effect
size until the total sample size in the output matched our projected
sample size (n=60). The protocol of the power analysis is here:

\emph{Input:}

Effect size f² = 0,24

α err prob = 0,05

Power (1-β err prob) = 0,7

Number of predictors = 8

\emph{Output:}

Noncentrality parameter λ = 14,4000000

Critical F = 2,1260234

Numerator df = 8

Denominator df = 51

Total sample size = 60

Actual power = 0,7039242

This means that, with our minimum sample size of 60, we have a 70\%
chance of detecting a small effect (approximated at
f\textsuperscript{2}=0.20 by Cohen 1988).

\emph{code not shown in pdf, see .rmd link above}

\textbf{H2: P2 - Space use behaviors vary among populations across the
range}

To roughly estimate our ability to detect actual effects, we ran a power
analysis in G*Power in the same way as for Hypothesis 1. The protocol of
the power analysis is here:

\emph{Input:}

Effect size f² = 0,18

α err prob = 0,05

Power (1-β err prob) = 0,7

Number of predictors = 4

\emph{Output:}

Noncentrality parameter λ = 10,6200000

Critical F = 2,5429175

Numerator df = 4

Denominator df = 54

Total sample size = 59

Actual power = 0,7025819

This means that, with our minimum sample size of 60, we have a 70\%
chance of detecting a small effect (approximated at
f\textsuperscript{2}=0.20 by Cohen 1988).

\emph{code not shown in pdf, see .rmd link above}

\hypertarget{e.-ethics}{%
\subsubsection{E. ETHICS}\label{e.-ethics}}

This research is carried out in accordance with permits from the:

\begin{enumerate}
\def\labelenumi{\arabic{enumi})}
\tightlist
\item
  US Fish and Wildlife Service (scientific collecting permit number
  MB76700A-0,1,2)
\item
  US Geological Survey Bird Banding Laboratory (federal bird banding
  permit number 23872)
\item
  Arizona Game and Fish Department (scientific collecting license number
  SP594338 {[}2017{]}, SP606267 {[}2018{]}, SP639866 {[}2019{]}, and
  SP402153 {[}2020{]})
\item
  Institutional Animal Care and Use Committee at Arizona State
  University (protocol number 17-1594R)
\end{enumerate}

\hypertarget{f.-author-contributions}{%
\subsubsection{F. AUTHOR CONTRIBUTIONS}\label{f.-author-contributions}}

\textbf{McCune:} Hypothesis development, data collection (trapping, GPS
tracking), data analysis and interpretation, write up, revising/editing.

\textbf{Folsom:} Data collection (trapping, GPS tracking),
revising/editing.

\textbf{Ross:} Model development, data analysis and interpretation,
revising/editing.

\textbf{Bergeron:} Data collection (trapping, GPS tracking),
revising/editing.

\textbf{Logan:} Hypothesis development, data interpretation, write up,
revising/editing, materials/funding.

\hypertarget{g.-funding}{%
\subsubsection{G. FUNDING}\label{g.-funding}}

This research is funded by the Department of Human Behavior, Ecology and
Culture at the Max Planck Institute for Evolutionary Anthropology.

\hypertarget{h.-conflict-of-interest-disclosure}{%
\subsubsection{H. CONFLICT OF INTEREST
DISCLOSURE}\label{h.-conflict-of-interest-disclosure}}

We, the authors, declare that we have no financial conflicts of interest
with the content of this article. Corina Logan is a Recommender and on
the Managing Board at PCI Ecology.

\hypertarget{i.-acknowledgements}{%
\subsubsection{I. ACKNOWLEDGEMENTS}\label{i.-acknowledgements}}

We thank Melissa Wilson Sayres for sponsoring our affiliations at
Arizona State University (ASU); Kevin Langergraber for serving as the
local PI on the ASU IACUC; Kristine Johnson for technical advice on
great-tailed grackles; Julia Cissewski and Sophie Kaube for tirelessly
solving problems involving financial transactions and contracts; Richard
McElreath for project support; and our research assistants: Aldora
Messinger, Elysia Mamola, Michael Guillen, Rita Barakat, Adriana
Boderash, Olateju Ojekunle, August Sevchik, Justin Huynh, Jennifer
Berens, Michael Pickett, Amanda Overholt, Emily Blackwell, Kaylee
Delcid, Brynna Hood, Sam Bowser, Sierra Planck, Samuel Munoz, Sawyer
Lung.

\hypertarget{j.-references}{%
\subsubsection*{\texorpdfstring{J.
\href{MyLibrary.bib}{REFERENCES}}{J. REFERENCES}}\label{j.-references}}
\addcontentsline{toc}{subsubsection}{J.
\href{MyLibrary.bib}{REFERENCES}}

\hypertarget{refs}{}
\leavevmode\hypertarget{ref-amundson2014hierarchical}{}%
Amundson CL, Royle JA, Handel CM. 2014. A hierarchical model combining
distance sampling and time removal to estimate detection probability
during avian point counts. The Auk: Ornithological Advances.
131(4):476--494.

\leavevmode\hypertarget{ref-blackburn2011proposed}{}%
Blackburn TM, Pyšek P, Bacher S, Carlton JT, Duncan RP, Jarošı́k V,
Wilson JR, Richardson DM. 2011. A proposed unified framework for
biological invasions. Trends in ecology \& evolution. 26(7):333--339.

\leavevmode\hypertarget{ref-bubb2006movement}{}%
Bubb DH, Thom TJ, Lucas MC. 2006. Movement, dispersal and refuge use of
co-occurring introduced and native crayfish. Freshwater Biology.
51(7):1359--1368.

\leavevmode\hypertarget{ref-butchart2010global}{}%
Butchart SH, Walpole M, Collen B, Van Strien A, Scharlemann JP, Almond
RE, Baillie JE, Bomhard B, Brown C, Bruno J, et al. 2010. Global
biodiversity: Indicators of recent declines. Science.
328(5982):1164--1168.

\leavevmode\hypertarget{ref-calabrese2016ctmm}{}%
Calabrese JM, Fleming CH, Gurarie E. 2016. Ctmm: An r package for
analyzing animal relocation data as a continuous-time stochastic
process. Methods in Ecology and Evolution. 7(9):1124--1132.

\leavevmode\hypertarget{ref-calenge2011home}{}%
Calenge C. 2011. Home range estimation in r: The adehabitatHR package.
Office national de la classe et de la faune sauvage: Saint Benoist,
Auffargis, France.

\leavevmode\hypertarget{ref-carere2013animal}{}%
Carere C, Gherardi F. 2013. Animal personalities matter for biological
invasions. Trends in ecology \& evolution. 1(28):5--6.

\leavevmode\hypertarget{ref-carpenter2017stan}{}%
Carpenter B, Gelman A, Hoffman MD, Lee D, Goodrich B, Betancourt M,
Brubaker M, Guo J, Li P, Riddell A. 2017. Stan: A probabilistic
programming language. Journal of statistical software. 76(1).

\leavevmode\hypertarget{ref-chapple2012can}{}%
Chapple DG, Simmonds SM, Wong BB. 2012. Can behavioral and personality
traits influence the success of unintentional species introductions?
Trends in ecology \& evolution. 27(1):57--64.

\leavevmode\hypertarget{ref-clavero2005invasive}{}%
Clavero M, Garcı́a-Berthou E. 2005. Invasive species are a leading cause
of animal extinctions. Trends in ecology \& evolution. 20(3):110.

\leavevmode\hypertarget{ref-cohen1988statistical}{}%
Cohen J. 1988. Statistical power analysis for the behavioral sciences
2nd edn.

\leavevmode\hypertarget{ref-cote2010personality}{}%
Cote J, Clobert J, Brodin T, Fogarty S, Sih A. 2010.
Personality-dependent dispersal: Characterization, ontogeny and
consequences for spatially structured populations. Philosophical
Transactions of the Royal Society B: Biological Sciences.
365(1560):4065--4076.

\leavevmode\hypertarget{ref-cushman2005elephants}{}%
Cushman SA, Chase M, Griffin C. 2005. Elephants in space and time.
Oikos. 109(2):331--341.

\leavevmode\hypertarget{ref-dingemanse2003natal}{}%
Dingemanse NJ, Both C, Van Noordwijk AJ, Rutten AL, Drent PJ. 2003.
Natal dispersal and personalities in great tits (parus major).
Proceedings of the Royal Society of London Series B: Biological
Sciences. 270(1516):741--747.

\leavevmode\hypertarget{ref-dray2010exploratory}{}%
Dray S, Royer-Carenzi M, Calenge C. 2010. The exploratory analysis of
autocorrelation in animal-movement studies. Ecological Research.
25(3):673--681.

\leavevmode\hypertarget{ref-duckworth2007coupling}{}%
Duckworth RA, Badyaev AV. 2007. Coupling of dispersal and aggression
facilitates the rapid range expansion of a passerine bird. Proceedings
of the National Academy of Sciences. 104(38):15017--15022.

\leavevmode\hypertarget{ref-faul2009statistical}{}%
Faul F, Erdfelder E, Buchner A, Lang A-G. 2009. Statistical power
analyses using g* power 3.1: Tests for correlation and regression
analyses. Behavior research methods. 41(4):1149--1160.

\leavevmode\hypertarget{ref-faul2007g}{}%
Faul F, Erdfelder E, Lang A-G, Buchner A. 2007. G* power 3: A flexible
statistical power analysis program for the social, behavioral, and
biomedical sciences. Behavior research methods. 39(2):175--191.

\leavevmode\hypertarget{ref-flockhart2016factors}{}%
Flockhart D, Mitchell G, Krikun R, Bayne E. 2016. Factors driving
territory size and breeding success in a threatened migratory songbird,
the canada warbler. Avian Conservation and Ecology. 11(2).

\leavevmode\hypertarget{ref-fogarty2011social}{}%
Fogarty S, Cote J, Sih A. 2011. Social personality polymorphism and the
spread of invasive species: A model. The American Naturalist.
177(3):273--287.

\leavevmode\hypertarget{ref-garabedian2018relative}{}%
Garabedian JE, Moorman CE, Peterson MN, Kilgo JC. 2018. Relative
importance of social factors, conspecific density, and forest structure
on space use by the endangered red-cockaded woodpecker: A new
consideration for habitat restoration. The Condor: Ornithological
Applications. 120(2):305--318.

\leavevmode\hypertarget{ref-glahn1997controlling}{}%
Glahn JF, Palacios JD, Garrison M. 1997. Controlling great-tailed
grackle damage to citrus in the lower rio grande valley, texas.

\leavevmode\hypertarget{ref-greenwood1982natal}{}%
Greenwood PJ, Harvey PH. 1982. The natal and breeding dispersal of
birds. Annual review of ecology and systematics. 13(1):1--21.

\leavevmode\hypertarget{ref-hayes2008there}{}%
Hayes KR, Barry SC. 2008. Are there any consistent predictors of
invasion success? Biological invasions. 10(4):483--506.

\leavevmode\hypertarget{ref-hertel2020guide}{}%
Hertel AG, Niemelä PT, Dingemanse NJ, Mueller T. 2020. A guide for
studying among-individual behavioral variation from movement data in the
wild. Movement Ecology. 8(1):1--18.

\leavevmode\hypertarget{ref-johnson2000male}{}%
Johnson K, DuVal E, Kielt M, Hughes C. 2000. Male mating strategies and
the mating system of great-tailed grackles. Behavioral Ecology.
11(2):132--141.

\leavevmode\hypertarget{ref-johnson2001great}{}%
Johnson K, Peer BD. 2001. Great-tailed grackle: Quiscalus mexicanus.
Birds of North America, Incorporated.

\leavevmode\hypertarget{ref-leo2016home}{}%
Leo BT, Anderson JJ, Phillips RB, Ha RR. 2016. Home range estimates of
feral cats (felis catus) on rota island and determining asymptotic
convergence1. Pacific science. 70(3):323--332.

\leavevmode\hypertarget{ref-lindstrom2013rapid}{}%
Lindström T, Brown GP, Sisson SA, Phillips BL, Shine R. 2013. Rapid
shifts in dispersal behavior on an expanding range edge. Proceedings of
the National Academy of Sciences. 110(33):13452--13456.

\leavevmode\hypertarget{ref-may2016predicting}{}%
May TM, Page MJ, Fleming PA. 2016. Predicting survivors: Animal
temperament and translocation. Behavioral Ecology. 27(4):969--977.

\leavevmode\hypertarget{ref-mettke2009spatial}{}%
Mettke-Hofmann C, Lorentzen S, Schlicht E, Schneider J, Werner F. 2009.
Spatial neophilia and spatial neophobia in resident and migratory
warblers (sylvia). Ethology. 115(5):482--492.

\leavevmode\hypertarget{ref-mettke2002significance}{}%
Mettke-Hofmann C, Winkler H, Leisler B. 2002. The significance of
ecological factors for exploration and neophobia in parrots. Ethology.
108(3):249--272.

\leavevmode\hypertarget{ref-minderman2010novel}{}%
Minderman J, Reid JM, Hughes M, Denny MJ, Hogg S, Evans PG, Whittingham
MJ. 2010. Novel environment exploration and home range size in starlings
sturnus vulgaris. Behavioral Ecology. 21(6):1321--1329.

\leavevmode\hypertarget{ref-mong2007optimizing}{}%
Mong TW, Sandercock BK. 2007. Optimizing radio retention and minimizing
radio impacts in a field study of upland sandpipers. The Journal of
wildlife management. 71(3):971--980.

\leavevmode\hypertarget{ref-nathan2008movement}{}%
Nathan R, Getz WM, Revilla E, Holyoak M, Kadmon R, Saltz D, Smouse PE.
2008. A movement ecology paradigm for unifying organismal movement
research. Proceedings of the National Academy of Sciences.
105(49):19052--19059.

\leavevmode\hypertarget{ref-nichols2009inferences}{}%
Nichols JD, Thomas L, Conn PB. 2009. Inferences about landbird abundance
from count data: Recent advances and future directions. In: Modeling
demographic processes in marked populations. Springer. pp. 201--235.

\leavevmode\hypertarget{ref-noonan2019comprehensive}{}%
Noonan MJ, Tucker MA, Fleming CH, Akre TS, Alberts SC, Ali AH, Altmann
J, Antunes PC, Belant JL, Beyer D, et al. 2019. A comprehensive analysis
of autocorrelation and bias in home range estimation. Ecological
Monographs. 89(2):e01344.

\leavevmode\hypertarget{ref-van2009personality}{}%
Overveld T van, Matthysen E. 2009. Personality predicts spatial
responses to food manipulations in free-ranging great tits (parus
major). Biology letters. 6(2):187--190.

\leavevmode\hypertarget{ref-pacheco2019nahua}{}%
Pacheco-Cobos L, Winterhalder B, Cuatianquiz-Lima C, Rosetti MF, Hudson
R, Ross CT. 2019. Nahua mushroom gatherers use area-restricted search
strategies that conform to marginal value theorem predictions.
Proceedings of the National Academy of Sciences. 116(21):10339--10347.

\leavevmode\hypertarget{ref-pebesma2018simple}{}%
Pebesma E. 2018. Simple features for r: Standardized support for spatial
vector data. The R Journal. 10(1):439--446.

\leavevmode\hypertarget{ref-peig2009new}{}%
Peig J, Green AJ. 2009. New perspectives for estimating body condition
from mass/length data: The scaled mass index as an alternative method.
Oikos. 118(12):1883--1891.

\leavevmode\hypertarget{ref-rappole1991new}{}%
Rappole JH, Tipton AR. 1991. New harness design for attachment of radio
transmitters to small passerines (nuevo diseño de arnés para atar
transmisores a passeriformes pequeños). Journal of field
Ornithology.:335--337.

\leavevmode\hypertarget{ref-rcoreteam}{}%
R Core Team. 2017. R: A language and environment for statistical
computing. Vienna, Austria: R Foundation for Statistical Computing.
\url{https://www.R-project.org}.

\leavevmode\hypertarget{ref-reale2007integrating}{}%
Réale D, Reader SM, Sol D, McDougall PT, Dingemanse NJ. 2007.
Integrating animal temperament within ecology and evolution. Biological
reviews. 82(2):291--318.

\leavevmode\hypertarget{ref-spiegel2016feedback}{}%
Spiegel O, Crofoot MC. 2016. The feedback between where we go and what
we know---information shapes movement, but movement also impacts
information acquisition. Current opinion in behavioral sciences.
12:90--96.

\leavevmode\hypertarget{ref-stuber2013slow}{}%
Stuber EF, Araya-Ajoy YG, Mathot KJ, Mutzel A, Nicolaus M, Wijmenga JJ,
Mueller JC, Dingemanse NJ. 2013. Slow explorers take less risk: A
problem of sampling bias in ecological studies. Behavioral Ecology.
24(5):1092--1098.

\leavevmode\hypertarget{ref-swihart1985testing}{}%
Swihart RK, Slade NA. 1985. Testing for independence of observations in
animal movements. Ecology. 66(4):1176--1184.

\leavevmode\hypertarget{ref-wehtje2003range}{}%
Wehtje W. 2003. The range expansion of the great-tailed grackle
(quiscalus mexicanus gmelin) in north america since 1880. Journal of
Biogeography. 30(10):1593--1607.

\leavevmode\hypertarget{ref-zucca2007piagetian}{}%
Zucca P, Milos N, Vallortigara G. 2007. Piagetian object permanence and
its development in eurasian jays (garrulus glandarius). Animal
Cognition. 10(2):243--258.

\end{document}
